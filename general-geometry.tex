\chapter{General geometry}

\begin{theorem}[Triangle inequality in $\R$]
    Let $a, b \in \R$, then the triangle inequality is defined:
    \begin{equation}
        |a + b| \le |a| + |b|
    \end{equation}
\end{theorem}
\begin{proof}
    Obviously: $|a| \ge a \wedge |b| \ge b \quad \forall a, b \in \R$
    
    Then, there two cases to consider, namely $a+b>0$ and $a+b<0$.

    \emph{Case $a+b>0$:}
    \begin{equation}
        |a + b| = a + b \le |a| + |b|
    \end{equation}

    \emph{Case $a+b<0$:}
    \begin{equation}
        |a + b| = -(a + b) = (-a) + (-b) \le |a| + |b|
    \end{equation}
\end{proof}

\begin{theorem}[Triangle inequality in $\R^n$]
    Let $a, b \in \R^n$, then the triangle inequality is defined:
    \begin{equation}
        |a + b| \le |a| + |b|
    \end{equation}
\end{theorem}
\begin{proof}
    The scalar product has the following property: $\langle a,b \rangle = |a||b|\cos(\sphericalangle a,b) \le |a||b|$
    \begin{equation}
        |a + b|^2 = \langle a+b,a+b \rangle 
                  = \langle a,a \rangle + 2 \langle a,b \rangle + \langle b,b \rangle
                  = |a|^2 + 2 \langle a,b \rangle + |b|^2
                  \le |a|^2 + |b|^2
    \end{equation}
\end{proof}

\begin{theorem}[Volume of a pyramid with arbitrary base area]
    The volume of pyramid with arbitrary base $A$ and height $h$ is:
    \begin{equation}
        V = \frac{1}{3}Ah
    \end{equation}
\end{theorem}
\begin{proof}
    By using the intercept theorem, one can show that an area $dA=dx * dy$ defined 
    by its infinitesimal width $dx$ and height $dy$ is scaled along the ray by 
    (denoted by $dA'$ with scaled dimensions $dx'$ and $dy'$):
    \begin{equation}
        dx' = s * dx \land dy' = s * dy \implies dA' = dx' * dy' = s^2 * dx * dy
    \end{equation}
    The scale along its height starting from the top can be expressed as:
    \begin{equation}
        s(x) = \frac{x}{h}
    \end{equation}
    Then, the cross-sectional area $A_c(x)$ at a given height is:
    \begin{equation}
        A_c(x) = A * s(x)^2 = A \frac{x^2}{h^2}
    \end{equation}
    The volume gets divided into $n$ slices at height $x = i \Delta h$ 
    each with a height of $\Delta h=i\frac{h}{n}$:
    \begin{equation}
        V_i = A_c(i \Delta h) * \Delta h 
            = \frac{A}{h^2} * \left(i * \frac{h}{n}\right)^2 * \frac{h}{n} 
            = \frac{A h}{n^3} i^2
    \end{equation}
    The total approximated volume is:
    \begin{equation}
        \begin{split}
            V & = \sum_{i}^{n}V_i 
                = \sum_{i}^{n}\frac{A h}{n^3} i^2
                =  \frac{A h}{n^3} \sum_{i}^{n}i^2 \\
              & =  \frac{A h}{n^3} \frac{n(n+1)(2n+1)}{6} \\
              & =  \frac{A h}{n^3} (2n^3+3n^2+n) \\
              & =  A\frac{h}{6} (2+\frac{3}{n}+\frac{1}{n^2}) \\
        \end{split}
    \end{equation}
    Then, by $n \rightarrow \infty$ the correct total volume is derived:
    \begin{equation}
        V = \lim_{n \rightarrow \infty}A\frac{h}{6} (2+\frac{3}{n}+\frac{1}{n^2}) = \frac{1}{3}Ah
    \end{equation}
\end{proof}

