\chapter{Sequences and series}

\begin{definition}[Harmonic series]
    The harmonic series is the sum of all fractions.
    \begin{equation}
        \sum_{n=1}^{\infty}\frac{1}{n} = 
            1 + 
            \frac{1}{2} + 
            \frac{1}{3} + \frac{1}{4} + 
            \frac{1}{5} + \frac{1}{6} + \frac{1}{7} + \frac{1}{8} + ...
    \end{equation}
\end{definition}

\begin{theorem}[Divergence of the harmonic series]
    The harmonic series is divergent. At one can right
    \begin{equation}
        \lim_{n\to\infty}\sum_{n=1}^{\infty}\frac{1}{n} = \infty
    \end{equation}
\end{theorem}
\begin{proof}
    The harmonic series can be written as
    \begin{equation}
        \sum_{n=1}^{\infty}\frac{1}{n} = 1 + \sum_{n=2}^{\infty}\frac{1}{n}
    \end{equation}
    Looking at the second term, one gets:
    \begin{equation}
        \sum_{n=2}^{\infty}\frac{1}{n} = 
            \frac{1}{2} + 
            \frac{1}{3} + \frac{1}{4} + 
            \frac{1}{5} + \frac{1}{6} + \frac{1}{7} + \frac{1}{8} + ...
    \end{equation}
    Obviously, this series gets bigger and bigger with every iteration.
    To proof divergence it is enough to show that a series exists which is always increasing 
    but smaller than the harmonic series.
    If it is possible to show divergence for the smaller series,
    then the harmonic series has to be divergent as-well.

    The idea to construct a series like described above, is, to always group fractions.
    A new group starts when the next $2^n$ fraction is reached.
    Then, all fraction within the group get the denominator replaced with the greatest denominator 
    of that group:
    \begin{equation}
        \begin{split}
            \sum_{n=2}^{\infty}\frac{1}{n} & = 
            \frac{1}{2}
            + \left(\frac{1}{3} + \frac{1}{4}\right) 
            + \left(\frac{1}{5} + \frac{1}{6} + \frac{1}{7} + \frac{1}{8}\right) + \\
            & \left(\frac{1}{9} + \frac{1}{10} + \frac{1}{11} + \frac{1}{12} + \frac{1}{13} + 
            \frac{1}{14} + \frac{1}{15} + \frac{1}{16}\right) + ... \\
            & \ge \frac{1}{2} 
                  + \left(\frac{1}{4} + \frac{1}{4}\right)
                  + \left(\frac{1}{8} + \frac{1}{8} + \frac{1}{8} + \frac{1}{8}\right) + \\
            &     \left(\frac{1}{16} + \frac{1}{16} + \frac{1}{16} + \frac{1}{16} + \frac{1}{16} + 
                          \frac{1}{16} + \frac{1}{16} + \frac{1}{16}\right) + ... \\
            & = \frac{1}{2} + \frac{1}{2} + \frac{1}{2} + \frac{1}{2} + ...
        \end{split}
    \end{equation}
    By doing this, the divergence becomes obvious. This approach can be further formalized:
    \begin{equation}
        \sum_{n=2}^{\infty}\frac{1}{n} = \sum_{n=0}^{\infty}\sum_{i=1}^{2^n}\frac{1}{2^n+i}
    \end{equation}
    Since 
    \begin{equation}
        2^{n+1} \ge 2^n+i \Rightarrow \frac{1}{2^n+i} \ge \frac{1}{2^{n+1}} \forall i \in \left[1...2^n\right], n \in \mathbb{N}
    \end{equation}
    one can conclude:
    \begin{equation}
        \sum_{n=2}^{\infty}\frac{1}{n} = \sum_{n=0}^{\infty}\sum_{i=1}^{2^n}\frac{1}{2^n+i}
            \ge \sum_{n=0}^{\infty}\sum_{i=1}^{2^n}\frac{1}{2^{n+1}}
            = \sum_{n=0}^{\infty}\frac{2^n}{2^{n+1}} = \sum_{n=0}^{\infty}\frac{1}{2} \to \infty
    \end{equation}
    The sum on the right side of the greater equal sign is increasing and always smaller than 
    left part.
    Yet, the right part is obviously divergent, then so is the left part.
    Therefore, the harmonic series is also divergent:
    \begin{equation}
        \sum_{n=1}^{\infty}\frac{1}{n} = 1 + \sum_{n=2}^{\infty}\frac{1}{n} \to \infty
    \end{equation}
\end{proof}