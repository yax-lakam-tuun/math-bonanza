\chapter{Spherical trigonometry}

All angles are in radians if not stated otherwise.

\begin{definition}[Great circle]
    Given a sphere with radius $r$, a great circle is a circle on the sphere's surface
    whose center coincides with the center of the sphere.
\end{definition}

\begin{theorem}[Two great circles intersect at two opposing points]
   Given a sphere with radius $r$ with its center $M$ and two great circles defined by 
   non-diametral surface points $A$ and $B$ and $P$, $Q$ respectively, both circles intersect at
   diametrically opposite points $X$ and $Y$.
   The line $\overline{XY}$ passes through the sphere's center $M$.
\end{theorem}
\begin{proof}
   Let $E_1$ define the plane for $A$, $B$ and $M$ and let $E_2$ define the place $P$, $Q$ and $M$.
   Both planes intersect since they have a common point $M$.
   The intersection is always a line which trivially runs through $M$ and pierces the sphere
   in two opposing points.
\end{proof}

\begin{definition}[Lune]
    Given a sphere with radius $r$, a lune, also called a digon, is the set of points bound
    by the intersection of two great circles.
    A lune is usually characterized by the internal angle both great circles shape.
    The word `lune' is derived from the Latin word \emph{luna} for moon.
\end{definition}

\begin{corollary}
    When two great circles intersect, one actually gets four lunes.
    The opposing lunes have the same internal angle $\phi$ and $\theta$ and $\phi+\theta$ 
    is always $2 \pi$.
\end{corollary}

\begin{lemma}
    Given a sphere with radius $r$, the area of a lune with internal angle $\phi$ 
    can be calculated with the following equation: $A = 2r^2\phi$.
\end{lemma}

\begin{proof}
    A lune covering a half-sphere has an internal angle of $\pi$.
    Since the sphere has a surface of $A=4 \pi r^2$, a half-sphere lune has an area of $2 \pi r^2$.
    A quarter-sphere has a surface of $A=\pi r^2$. And so on.
    The relationship between a lune's internal angle $\phi$ and its area has a linear relationship
    and can be written as:
    \begin{equation}
        A = 4 \pi r^2 \frac{\phi}{2\pi} = 2 r^2 \phi
    \end{equation}
\end{proof}

\begin{definition}[Spherical triangle (Euler triangle)]
    Given a sphere with radius \emph{r} and three points $P$, $Q$, and $R$ on the sphere's surface 
    which do not reside on one single great circle, these three points span three 
    different great circles namely $PQ$, $PR$ and $QR$.
    Therefore, there are six intersection points $A$, $B$, $C$, $A'$, $B'$ and $C'$.
    The points $A'$, $B'$ and $C'$ represent the counterparts of $A$, $B$ and $C$.
    An angle inside a spherical triangle corresponds to the angle created by both tangents 
    at the vertex or the angle between the planes of the great circles respectively.
    Without any further restriction, there would be many possible combinations of triangles.
    Euler triangles demand: $\alpha, \beta, \gamma \le \pi$.
    Then, one gets two opposing spherical triangles.
\end{definition}
